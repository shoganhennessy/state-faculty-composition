%%%%%%%%%%%%%%%%%%%%%%%%%%%%%%%%%%%%%%%%%
%% Section on possible outcomes
\section{Conceptual Framework}
\label{sec:conceptual}
%
%\textbf{Follow Prof Riehl's comments.
%    Also consider a theory toy-model where reach and teaching are plasuible substitutes, following \cite[Sec.~6.2]{NBERc13879}.
%    Better motivated if research spending is affected.
%}
%The, revenue shock induced, shift away from tenured profs may be mechanically relate to the shift between departments (towards STEM).
%Find a way to mix the “non-replacement of profs” with that for “department composition shift.”
%
%\subsection{University Responses to Financial Shocks}
%\label{sec:responses}

While state support for public higher education has stagnated at the same time as education costs rose, there are multiple possible ways that a university can respond.
\cite{NBERw23736} established that corresponding rises in tuition did not offset the falling state support, so that university spending (per student) fell in response to these persistent, negative state funding shocks.
There are multiple ways that these changes in finances may affect the faculty composition at public universities.

Universities every year hire a number of new professors, either to expand their departments or to replace leaving professors.
The number of professors hired is usually highest among non-tenured adjunct faculty, as these instructors are by majority hired on short-term contingent contracts, so that they are less costly to hire (or to not reinstate) in response to yearly changes in teaching needs.
Secondly, tenure-track assistant professors are hired in most years at public universities to replace leaving assistant professors, and are most often hired with a four to six year contract and agreement to be considered for tenure at the end of this term.
Lastly, tenured professors are those that have undergone review and have successfully secured a full-time appointment at their university with no expiration date; tenured professors may be hired from outside a university, though this position has the lowest hiring rate among most universities.
It is common for universities to restrict their faculty hiring in the face of financial shocks.
Since the yearly hiring rate is highest among lecturers and assistant professors, financial shocks may lower the number of lecturer and assistant professors at public universities the most, if faculty hiring is affected by falls in state funding.

Additionally, professors' salaries may be disrupted by the stagnation in public university finances.
When the university has a lower budget for its yearly hiring, it may respond by lowering the salary they offer to new hires.
This possible effect may not be the same across each position of professor; tenured faculty are often hired away from another university, so that new tenured professor hires could be less likely to accept a lower offer from a public university, and we would see that new tenured hires are not affected salary-wise.\footnote{
    \cite{blackaby2005} describes the academic outside options among economist faculty, documenting how they differ for academic position and gender.
}
Yet salary for all the professors, not just new hires, may also be affected: multiple universities passed a university-wide pay-cut for their faculty in response to state budget cuts around the 2008 recession, for example.\footnote{
    Indeed, Cornell University implemented both hiring freezes and even a nominal salary reduction for professors, in anticipation of a financial shock in early 2021.
    The salary cut was not permanent, as the oncoming financial shock turned out to not be as serious as projected, so that the salary cut was returned to professors in the year 2021.
}
Faculty are not paid the same by position, so that their are financial consequences for faculty being promoted between positions within a university.
In response to tightening fiscal restraints, universities may be less likely to grant tenure to assistant professors, or to promote tenured professors from associate to full professor either.

It is not immediately clear which effect will dominate, and which position of professor will be effected the most by changes in their university's finances.
Yet, there is one empirical fact worth noting: a mean assistant or tenured professor earns more than double that of an adjunct lecturer per year (\autoref{tab:illinois-summary}), and the tenure contract means universities are compelled to pay that higher salary for more years than contingent lecturers.
If a public university's primary obligation is to teach, and they must fulfil this objective with fewer and fewer resources, then they may substitute away from tenure-track and tenured professor towards contingent lecturers to achieve this obligation.\footnote{
    That is substituted away from tenure-track and tenured professors according to the salary ratio, and the relative productivity between these two positions.
    It is not possible to estimate the relative productivity in this paper, so I leave the rate of substitution between the two positions to further research.
}
