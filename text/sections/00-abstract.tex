\noindent
State support for higher education has stagnated on a per-student level, while public universities have reduced their employment of tenure-track or tenured professors, and increased their employment of contingent lecturers (per student).
This analysis relates these changes in the faculty composition at public universities to the fall in per-student university revenues from state appropriations.
A shift-share instrumental variables approach addresses endogeneity in decisions for state appropriations to higher education, by exploiting the yearly change in state support with an institution's reliance on that support.
A decrease in state funding of 10\%, via a shock to state appropriations, decreases the number of assistant professors per student at a public university by 1.4\% and full professors by 1.2\%, yet increases the number of lecturers per student by 4.4\%.
Local projection estimates show that these effects linger for up to three years after the initial shock.
Analysis of all the professors hired at Illinois public universities 2011-2021 shows that incumbent professors are not affected by the changes in state support, implying that these changes in faculty composition arose by reduced hiring of tenure-track and tenured professors at public universities.
These results exhibit the long-term effects of stagnating state support for higher education, and raises questions about the direction that public education heads as these financial headwinds show no sign of dissipating.

\vspace{0.75cm}
\noindent\textbf{Keywords:}
State and Local Budget and Expenditures,
Higher Education,
Public Sector Labor Markets

\vspace{0.5cm}
\noindent\textbf{JEL Codes:} H72, I23, J45
