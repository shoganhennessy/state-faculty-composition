\noindent
Public universities educate the majority of higher education students in the US, yet have experienced a secular decline in state funding (per student) the last three decades.
This decline has been shown to lead to worse education and later-life outcomes for students \citep{NBERw23736,NBERw27885}, yet it is not clear what mechanisms drive these effects, or how public universities are affected as institutions.
I show that four-year, degree-granting public universities have experienced falling state funding per student, and this affected their composition of faculty.
Falls in public university revenues from the state, instrumented by state-level finance shocks, lead to a fall in the number of tenure-track or tenured professors per student within a university, an increase in the number of lecturers, and overall fall in the number of faculty per student.
US private universities were not exposed to such financial constraints during this same time period, and do not exhibit the substitution away from tenure-track and tenured professors, so that the stagnating state support for public universities has implications for the wider structure of higher education instruction and research in the US.

US universities are widely considered the highest performing in the world, yet there are consequential differences between its universities that operate in the private sector and those established by state governments.
Public universities are subject to numerous state-level administrative laws, and rely on their state governments for funding: an average public university received around \$11,600 per enrolled student in 1990, yet only \$8,300 per enrolled student in 2021.
This fall is driven by stagnating funding provided by state governments, while enrolment among public universities rose by 46\% over the same time period.
At the same time, the number of professors per student at public universities stayed relatively stable at 0.045, yet fell from around 0.04 to around 0.35 professors per student at public universities, driven by a fall of over 20\% in the ratio of associate and full professors per student at public universities.

I use a variant of the \cite{NBERw23736,NBERw27885} shift-share instrument for state funding from shocks to appropriations for the entire state to identify the effect on the faculty composition.
State appropriations for public universities are plausibly endogenous to many outcomes: state governments decide on yearly budgets for their higher education sector, and this process can be influenced by local financial conditions, or even public perceptions of the state's higher education system.
The shift-share instrument interacts reliance on state funding in a base period with the yearly total appropriations per student in the entire state, to exploit both changes in support for higher education and how much each university relies on that support.

Findings show that negative changes in state funding affects faculty composition at public universities, away from tenured and tenure-track professors towards non-tenure track lecturer positions.
A reduction of 10\% in state funding, via a shock to state appropriations, leads to a fall in 1.4\% in the number of assistant professors per student within a university, 1.2\% for full (tenured) professors, and an increase in 4.4\% the number of lecturers per student.
Local projection methods show that these effects linger for a small number of years after the initial revenue shock.
Over this time period, state funding fell by around 35\%, while the count of professors per student fell by 9\%, so that these results show that falls in state funding explain around a third of the observed shift away from tenure-track and tenure professors towards contingent faculty.

Additionally, I use individual level data on professors at Illinois public universities over 2010-2021 to investigate whether the shocks to state funding affected incumbent professors at public universities.
Incumbent professors were not meaningfully affected, in terms of total salary, promotion rate, or rate of leaving the Illinois public university system, yet lecturers' first year salary decreased by 3.8\% in response to a fall in state funding of 10\%.
These findings imply that changes in faculty composition came about by changes in the composition of hiring new professors at the university level.

\cite{NBERw23736} first study the effect of (changes in) public universities' finances on student outcomes by isolating changes in public universities' expenditures on student instruction, addressing endogeneity of expenditures decisions by instrumenting expenditures with state appropriation shocks and limits on tuition price.
\cite{NBERw23736} find that increases in public university spending (via appropriation shocks) increases enrolment and degree completion among students, but corresponding changes in attendance price (via limits on tuition price) do not have an effect.\footnote{
    \cite{miller2022making} further analyse the effects of falls in university revenues, by finding that reductions in public universities tuition prices (suggestively motivated by falling state funding) leads to reductions in provided financial aid.
}
\cite{chakrabarti2018effect,NBERw27885} use the same instrument for public university revenues at the state level, combined with rich information on students' individual-level outcomes, to show that increases in state appropriations lead to degree lower completion time and later-life student debt.
\cite{bound2019public} use another variant of the same approach to show that state-level higher education spending cuts induced public universities to shift toward tuition as their primary source of revenue, and shift institutional resources towards students who pay higher tuition rates.
In a similar vein, \cite{bound2007cohort} document variation in per student state funding for higher education resulting from yearly changes in a state's student-age population; they find that lower per student funding lead to lower higher education completion rate among students.

Faculty are a core component of this country's educational and research-innovation university system, yet their composition within universities, and its relationship with higher education finances has so far not been studied.
\cite{brown2014endowment} presents the closest example by studying how university endowments react to negative financial shocks, exhibiting an association between negative endowment shocks and a fall in tenure-system faculty employed at private universities.
\cite{abe2015implications} present a model for decision-making by university administrators, and posit that a large component of variation in faculty salaries arises from university politics, and not only economic factors such as outside options.
\cite{johnson2009jep,NBERc13879} both empirically document allocation of faculty between research and instruction, and \cite{hemelt2021math} document differences in instructional cost (per student) between departments, primarily thanks to department differences in class size and faculty salaries.
\cite{turner2014impact} document how universities reacted to the Great Recession of 2008, a severe financial shock to state finances and support for higher education.

The number of professors, both in absolute terms and relative to number of students, may be particularly important for quality of instruction and thus educational outcomes.
\cite{angrist1999using} first causally show that reducing class size (via religious ruling) induces an increase in test scores for school-age children, yet the magnitude of this effect is not as clear in the higher education setting.
\cite{bandiera2010heterogeneous} use a fixed effect model, combined with multiple individual observations in a long panel, to find that UK university students perform worse academically in particularly large classes, and the difference is largest for students at the top of the academic distribution.
For the composition of faculty teaching university classes, \cite{bettinger2010does,figlio2015tenure} find US students have better enrolment and learning outcomes from courses taught by contingent or adjunct (i.e. not tenured or on tenure-track) professors (relative to tenured or tenure-track).
\cite{ehrenberg2005tenured} observe a negative association between utilisation of non-tenured instructors and student graduation rates.
My approach focuses on count of professors per student at universities, so does not directly observe individual class-sizes, yet presents considers this a possible mechanism through which stagnating state support for higher education negatively affects student outcomes \citep{NBERw23736,NBERw27885}.
